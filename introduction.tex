\chapter{序論}

本論文は、修士論文の書き方 \cite{taro} の一例を示す。


\section{本研究の位置付け}

ここでは、色々なサンプルを示す。
次の式 (\ref{eq:eq_1_1}) の通り $n$ 次元の超球を仮定する。$n = 3$ の場合は図\ref{fig:ball} のようになる。
\begin{align}
	r^2 = \sum_{k=1}^n x_k^2
	\label{eq:eq_1_1}
\end{align}
\begin{figure}[h]
	\centering
	\includegraphics[bb=0 0 79 65]{./figures/ball.png}
	\caption{3 次元の球}
	\label{fig:ball}
\end{figure}

一方で、表\ref{table:elem}によれば、a,b,c,d の 4 つの要素がある。
\begin{table}[h]
	\centering
	\renewcommand{\tablename}{表}
	\begin{tabular}{|c|c|}
		\hline
		a & b \\ \hline
		c & d \\ \hline
	\end{tabular}
	\caption{要素群}
	\label{table:elem}
\end{table}


